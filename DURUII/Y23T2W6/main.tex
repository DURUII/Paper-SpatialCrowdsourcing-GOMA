\documentclass[a4paper,10pt]{ctexart}

\usepackage{geometry}
\usepackage{booktabs}
\usepackage{graphicx}
\usepackage[final]{pdfpages}
\usepackage[stable]{footmisc}
\usepackage{threeparttable}
\usepackage{indentfirst}
\usepackage{minted}
\usepackage{listings}
\usepackage{xcolor}
\usepackage{subfigure}
\usepackage{amsmath}
\usepackage{amsfonts}

% \setlength{\parindent}{0pt}
\geometry{top=20mm,bottom=20mm,left=20mm,right=20mm}
\lstset{
    rulesepcolor= \color{gray},
    breaklines=true,
    numbers=left,
    numberstyle= \small,
    commentstyle=\color{gray},
    frame=shadowbox
}

\title{Y23T2W4 例行周报}
\author{杜睿}
% \date{}

\begin{document}
\tableofcontents
\maketitle

% \begin{abstract}
% \end{abstract}

\section{时空众包平台中的在线二维匹配}

基于童咏昕于TKDE 2021 论文及ADL 103所作报告。

\subsection{引言}

时空众包核心问题是任务分配,即在满足时空等条件约束下,如何为不同任务分配合适的工人,使得任务匹配个数(效用值)最大化。在线场景下,工人和任务都是实时出现的。

\subsection{title}

\section{在线匹配问题的近况与进展}

基于黄志毅于IJTCS-FAW 2022所作报告。


\section{数据驱动的拍卖机制设计}

基于黄志毅于2023年9月18日所作报告。

设计最优拍卖机制是微观经济学中一个非常重要的课题, 特别是网络经济的发展让这个课题更加具有现实意义。经济学中关于这个课题最重要的结果就是著名的Myerson 最优拍卖理论, Myerson 因此而获得了诺贝尔经济学奖 。但这个漂亮的经济学理论在现实中的使用却非常少,主要有两个原因: 一是这个最优拍卖机制比较复杂; 二是这个机制的最优性有很严格的数学假设条件,这些条件不一定在现实中满足。针对这些问题, 最近十几年里在理论计算机界有 一系列关于最有拍卖机制设计的工作, 主要突出机制的简单性与鲁棒性。本报告会综述这方面的工作并展望未来的方向 。

\subsection{基础模型与理论}

\subsection{统计学习理论取得的进展}

\subsection{乘积分布的可学习性与强收益单调性}

\end{document}

% \begin{table}[htbp]
%     \caption{常用的处理机调度策略}
%     \centering

%     \begin{threeparttable}
%         \begin{tabular}{cccccc}
%             \toprule
%             算法名称           & 主要适用范围       & 默认调度方式         \\
%             \midrule
%             先来先服务         & 作业调度\&进程调度 & 非抢占式             \\
%             短作业(进程)优先 & 作业调度\&进程调度 & 非抢占式             \\
%             高响应比优先       & 作业调度           & 非抢占式             \\
%             时间片轮转         & 进程调度           & 抢占式(不抢时间片) \\
%             多级反馈队列       & 进程调度           & 抢占式(抢占时间片) \\
%             \bottomrule
%         \end{tabular}

%         \zihao{-6}
%         \begin{tablenotes}
%             \item [*]   调度策略也就是调度算法
%         \end{tablenotes}

%     \end{threeparttable}
%     \qquad
% \end{table}

% \begin{figure}[htbp]
%     \centering
%     \includegraphics[height=550pt]{v1-class-compat.png}
%     \caption{UML类图(第二版)}
% \end{figure}

% \begin{minted}[mathescape,
%     linenos,
%     numbersep=5pt,
%     frame=lines,
%     gobble=4,
%     framesep=2mm]{Java}
%     public interface Observable {
%         void attachObserver(Observer o);

%         void detachObserver(Observer o);

%         void notifyObservers();
%     }
% \end{minted}

% \begin{lstlisting}[language={java},caption={收容队列(基于响应比的优先队列)}]
% private PriorityQueue<Task> queue = new PriorityQueue<>(new Comparator<Task>() {
%     @Override
%     public int compare(Task o1, Task o2) {
%         return (o2.getResponseRate(Clock.minutes) - o1.getResponseRate(Clock.minutes) > 0) ? (1) : (-1);
%     }
% });
% \end{lstlisting}
